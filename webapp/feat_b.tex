%!TEX root = robots_desc.tex

%  _____          _                    _____      _     ____
% |  ___|__  __ _| |_ _   _ _ __ ___  | ____|_  _| |_  | __ )
% | |_ / _ \/ _` | __| | | | '__/ _ \ |  _| \ \/ / __| |  _ \
% |  _|  __/ (_| | |_| |_| | | |  __/ | |___ >  <| |_  | |_) |
% |_|  \___|\__,_|\__|\__,_|_|  \___| |_____/_/\_\\__| |____/
%

\setSpacing{1.5}
\section{Feature Extension: B}

\noindent
เราจะเพิ่ม Feature ให้แก่ Web Service Application ดังนี้
\begin{itemize}[topsep=0pc,itemsep=0pc]
\item  
    สามารถกำหนดได้ว่าจะใช้ Distance Metric ใดในการวัดระยะห่างระหว่างจุดสองจุดของ Endpoint \npt{POST}{/distance} ได้
\end{itemize}

\setSpacing{1.5}
\subsection{Modified Changes}

\begin{itemize}[parsep=0.5pc]
\item 
    ผู้เข้าแข่งขันจะต้องแก้ไข Endpoint \npt{POST}{/distance} เพื่อให้รองรับ JSON Field ใหม่ชื่อว่า \lstinline{"metric"} ซึ่งจะเป็นสตริงได้เพียง 2 แบบคือ \lstinline{"euclidean"} (default) หรือ \lstinline{"manhattan"}

    ในกรณีที่เป็นค่าสตริง \lstinline{"manhattan"} ค่าระยะห่างระหว่างจุดสองจุดใน Input Data จะต้องถูกคำนวณเป็น Manhattan Distance แทนที่จะเป็น Euclidean Distance
\end{itemize}

\noindent
รายละเอียดเพิ่มเติมสามารถหาอ่านได้จากไฟล์ API Specification ที่แนบมากับเอกสารนี้

\setSpacing{1.5}
\subsection{Example Requests and Responses}

\noindent
พิจารณาลำดับของการเรียก API Endpoints ต่อไปนี้

\begin{itemize}
\item  %%%%%%%%%%%%%%%%%%%%%%%%%%%%%%%%%%
\textbf{Request \#1:}
\begin{lstlisting}[xleftmargin=1pc,numbers=none]
POST /distance HTTP/1.1
Content-Type: application/json

{
  "first_pos": {"x": 3, "y": -2},
  "second_pos": {"x": -1, "y": 1}
}
\end{lstlisting}
\textbf{Response \#1:}
\begin{lstlisting}[xleftmargin=1pc,numbers=none]
HTTP/1.1 200 OK
Content-Type: application/json

{
  "distance": 5.000
}
\end{lstlisting}

\item  %%%%%%%%%%%%%%%%%%%%%%%%%%%%%%%%%%
\textbf{Request \#2:}
\begin{lstlisting}[xleftmargin=1pc,numbers=none]
POST /distance HTTP/1.1
Content-Type: application/json

{
  "first_pos": {"x": 3, "y": -2},
  "second_pos": {"x": -1, "y": 1},
  "metric": "manhattan"
}
\end{lstlisting}
\newpage
\textbf{Response \#2:}
\begin{lstlisting}[xleftmargin=1pc,numbers=none]
HTTP/1.1 200 OK
Content-Type: application/json

{
  "distance": 7.000
}
\end{lstlisting}

\item  %%%%%%%%%%%%%%%%%%%%%%%%%%%%%%%%%%
\textbf{Request \#3:}
\begin{lstlisting}[xleftmargin=1pc,numbers=none]
POST /distance HTTP/1.1
Content-Type: application/json

{
  "first_pos": {"x": 3, "y": -2},
  "second_pos": {"x": -1, "y": 1},
  "metric": "euclidean"
}
\end{lstlisting}
\textbf{Response \#3:}
\begin{lstlisting}[xleftmargin=1pc,numbers=none]
HTTP/1.1 200 OK
Content-Type: application/json

{
  "distance": 5.000
}
\end{lstlisting}
\end{itemize}
