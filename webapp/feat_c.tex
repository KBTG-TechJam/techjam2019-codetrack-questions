%!TEX root = robots_desc.tex

%  _____          _                    _____      _      ____
% |  ___|__  __ _| |_ _   _ _ __ ___  | ____|_  _| |_   / ___|
% | |_ / _ \/ _` | __| | | | '__/ _ \ |  _| \ \/ / __| | |
% |  _|  __/ (_| | |_| |_| | | |  __/ | |___ >  <| |_  | |___
% |_|  \___|\__,_|\__|\__,_|_|  \___| |_____/_/\_\\__|  \____|
%

\setSpacing{1.5}
\section{Feature Extension: C}

\begin{quote}
  \footnotesize
  \textbf{หมายเหตุ:} Web Service Application ของผู้เข้าแข่งขันจำเป็นต้อง Implement Feature Extension A ก่อนจึงจะ Implement Feature ใหม่ต่อไปนี้ได้
\end{quote}

\noindent
เราจะเพิ่ม Feature ให้แก่ Web Service Application ดังนี้
\begin{itemize}[topsep=0pc,itemsep=0pc]
\item  
    Operator ที่เป็นมนุษย์สามารถ Query สอบถามได้ว่าจากจุดที่กำหนดให้ Robot ตัวใดอยู่ใกล้เคียงจุดดังกล่าวมากที่สุด (ภายใต้ Euclidean Distance Metric)
    โดยอ้างอิงจากข้อมูลตำแหน่งล่าสุดที่ Robot แต่ละตัวอัปเดตเข้ามากับ Service กลาง \;
    หากมี Robot หลายตัวที่มีระยะห่างจากจุดหนึ่งเท่ากัน ให้พิจารณา Robot ที่มีค่า ID น้อยกว่าก่อนเป็นอันดับแรก
\end{itemize}

\setSpacing{1.5}
\subsection{API Endpoints}

\noindent
สำหรับ Feature Extension นี้จะต้อง Implement HTTP Endpoint เพิ่มเติมอีก 1 จุดคือ

\begin{description}[parsep=0.5pc]
\item[\npt{POST}{/nearest}] ~ \\
    Query เพื่อเรียกดู Robot ตัวที่อยู่ใกล้กับจุดที่กำหนดให้มากที่สุด 
    
    โดย Request Object จะมี JSON Field \lstinline{"ref_position"} ซึ่งเป็นข้อมูลของจุดอ้างอิงที่กำหนดให้ ในรูปของพิกัด $(x, y) \in \mathbb{Z}^2$ ที่เป็น Object ที่มี 2 Subfield ซึ่งได้แก่ \lstinline{"x"} และ \lstinline{"y"}
    
    ส่วน Response Object จะประกอบไปด้วย JSON Field \lstinline{"robot_ids"} ซึ่งเป็นอาร์เรย์ของ Robot ID ที่ค้นพบจำนวน 0 หรือ 1 ตัว
    (อาร์เรย์อาจจะว่างเปล่าได้ หากไม่เคยมี Robot ตัวใดที่อัปเดตข้อมูลตำแหน่งของตนเองเลย)
\end{description}

\setSpacing{1.5}
\subsection{Example Requests and Responses}

\noindent
พิจารณาลำดับของการเรียก API Endpoints ต่อไปนี้

\begin{itemize}
\item  %%%%%%%%%%%%%%%%%%%%%%%%%%%%%%%%%%
\textbf{Request \#1:}
\begin{lstlisting}[xleftmargin=1pc,numbers=none]
POST /nearest HTTP/1.1
Content-Type: application/json

{
  "ref_position": {"x": -1, "y": 1}
}
\end{lstlisting}
\textbf{Response \#1:}
\begin{lstlisting}[xleftmargin=1pc,numbers=none]
HTTP/1.1 200 OK
Content-Type: application/json

{
  "robot_ids": []
}
\end{lstlisting}

\newpage
\item  %%%%%%%%%%%%%%%%%%%%%%%%%%%%%%%%%%
\textbf{Request \#2:}
\begin{lstlisting}[xleftmargin=1pc,numbers=none]
PUT /robot/1/position HTTP/1.1
Content-Type: application/json

{
  "position": {"x": 1, "y": 1}
}
\end{lstlisting}
\textbf{Response \#2:}
\begin{lstlisting}[xleftmargin=1pc,numbers=none]
HTTP/1.1 204 No Content
\end{lstlisting}

\item  %%%%%%%%%%%%%%%%%%%%%%%%%%%%%%%%%%
\textbf{Request \#3:}
\begin{lstlisting}[xleftmargin=1pc,numbers=none]
PUT /robot/2/position HTTP/1.1
Content-Type: application/json

{
  "position": {"x": 0, "y": 0}
}
\end{lstlisting}
\textbf{Response \#3:}
\begin{lstlisting}[xleftmargin=1pc,numbers=none]
HTTP/1.1 204 No Content
\end{lstlisting}


\item  %%%%%%%%%%%%%%%%%%%%%%%%%%%%%%%%%%
\textbf{Request \#4:}
\begin{lstlisting}[xleftmargin=1pc,numbers=none]
POST /nearest HTTP/1.1
Content-Type: application/json

{
  "ref_position": {"x": -1, "y": 1}
}
\end{lstlisting}
\textbf{Response \#4:}
\begin{lstlisting}[xleftmargin=1pc,numbers=none]
HTTP/1.1 200 OK
Content-Type: application/json

{
  "robot_ids": [2]
}
\end{lstlisting}

\item  %%%%%%%%%%%%%%%%%%%%%%%%%%%%%%%%%%
\textbf{Request \#5:}
\begin{lstlisting}[xleftmargin=1pc,numbers=none]
POST /nearest HTTP/1.1
Content-Type: application/json

{
  "ref_position": {"x": 1, "y": 0}
}
\end{lstlisting}
\newpage
\textbf{Response \#5:}
\begin{lstlisting}[xleftmargin=1pc,numbers=none]
HTTP/1.1 200 OK
Content-Type: application/json

{
  "robot_ids": [1]
}
\end{lstlisting}
\end{itemize}
