%!TEX root = robots_desc.tex

%  _____          _                    _____      _     ____
% |  ___|__  __ _| |_ _   _ _ __ ___  | ____|_  _| |_  |  _ \
% | |_ / _ \/ _` | __| | | | '__/ _ \ |  _| \ \/ / __| | | | |
% |  _|  __/ (_| | |_| |_| | | |  __/ | |___ >  <| |_  | |_| |
% |_|  \___|\__,_|\__|\__,_|_|  \___| |_____/_/\_\\__| |____/
%

\setSpacing{1.5}
\section{Feature Extension: D}

\begin{quote}
  \footnotesize
  \textbf{หมายเหตุ:} Web Service Application ของผู้เข้าแข่งขันจำเป็นต้อง Implement Feature Extension A ก่อนจึงจะ Implement Feature ใหม่ต่อไปนี้ได้
\end{quote}

\noindent
เราจะเพิ่ม Feature ให้แก่ Web Service Application ดังนี้
\begin{itemize}[topsep=0pc,itemsep=0pc]
\item  
    เมื่อ Robot ใช้คลื่นเรดาร์ตรวจพบวัตถุต่างดาวพบ (เช่น เศษของสะเก็ดดาวตก) แล้ว Robot ดังกล่าวจะสามารถรายงานข้อมูลดังกล่าวมายัง Web Service กลางได้ \;
    ข้อมูลดังกล่าวจะประกอบไปด้วย Object DNA (ซึ่งเป็นสตริงที่ประกอบไปด้วย \lstinline{'a'}–\lstinline{'z'} ความยาว 16 อักษร และสามารถ Uniquely Identify วัตถุที่เป็น Alien Object ได้) และข้อมูลระยะห่างระหว่าง Robot ที่ทำการสำรวจและวัตถุดังกล่าว (ภายใต้ Euclidean Distance Metric)
\item 
    Operator ที่เป็นมนุษย์สามารถ Query เพื่อสอบถามว่าวัตถุต่างดาวหนึ่ง ๆ น่าจะอยู่ในตำแหน่งใดพื้นที่ โดยอาศัยข้อมูลเรดาร์ที่ Robot รายงานเข้ามา \; โดยให้ตอบค่าพิกัดเป็นเศษเป็นจำนวนเต็ม $(x, y) \in \mathbb{Z}^2$
\end{itemize}

\setSpacing{1.5}
\subsection{API Endpoints}

\noindent
สำหรับ Feature Extension นี้จะต้อง Implement HTTP Endpoint เพิ่มเติมอีก 2 จุดคือ

\begin{description}[parsep=0.5pc]
\item[\npt{POST}{/alien/\{object\_dna\}/report}] ~ \\
    Update ข้อมูลการใช้คลื่นเรดาร์กับวัตถุต่างดาว \;
    โดย Path Parameter \lstinline{object_dna} จะระบุ Object DNA ของวัตถุต่างดาวดังกล่าว และข้อมูลจาก JSON Field \lstinline{"robot_id"} จะระบุ ID ของ Robot ที่กำลังรายงานผล และ JSON Field \lstinline{"distance"} จะระบุระยะห่างระหว่าง Robot และวัตถุต่างดาวดังกล่าว
\item[\npt{GET}{/alien/\{object\_dna\}/position}] ~ \\
    Query เพื่อให้ Web Service กลางช่วยคำนวณหาตำแหน่งที่ถูกต้องแน่นอนของวัตถุต่างดาวที่กำหนดให้ \; 
    โดย Path Parameter \lstinline{object_dna} จะระบุ Object DNA ของวัตถุต่างดาวข้างต้น

    หากสามารถคำนวณตำแหน่งของวัตถุต่างดาวได้อย่างเจาะจง จะต้องคืน Status 200 OK พร้อมกับ Response Object ที่มี JSON Field \lstinline{"position"} ที่ระบุตำแหน่งของวัตถุต่างดาวที่คำนวณได้ \;
    หากข้อมูลที่มีอยู่ไม่เพียงพอที่จะคำนวณตำแหน่งที่แน่นอนได้ในเชิงเรขาคณิต จะต้องคืน Status 424 Failed Dependency แทน
\end{description}

\noindent        
รายละเอียดเพิ่มเติมสามารถหาอ่านได้จากไฟล์ API Specification ที่แนบมากับเอกสารนี้

\setSpacing{1.5}
\subsection{Example Requests and Responses}

\noindent
พิจารณาลำดับของการเรียก API Endpoints ต่อไปนี้

\newpage
\begin{itemize}
\item  %%%%%%%%%%%%%%%%%%%%%%%%%%%%%%%%%%
\textbf{Request \#1:}
\begin{lstlisting}[xleftmargin=1pc,numbers=none]
PUT /robot/1/position HTTP/1.1
Content-Type: application/json

{
  "position": {"x": 0, "y": 0}
}
\end{lstlisting}
\textbf{Response \#1:}
\begin{lstlisting}[xleftmargin=1pc,numbers=none]
HTTP/1.1 204 No Content
\end{lstlisting}

\item  %%%%%%%%%%%%%%%%%%%%%%%%%%%%%%%%%%
\textbf{Request \#2:}
\begin{lstlisting}[xleftmargin=1pc,numbers=none]
POST /alien/abcdefghijklmnop/report HTTP/1.1
Content-Type: application/json

{
  "robot_id": 1,
  "distance": 3.162
}
\end{lstlisting}
\textbf{Response \#2:}
\begin{lstlisting}[xleftmargin=1pc,numbers=none]
HTTP/1.1 200 OK
\end{lstlisting}

\item  %%%%%%%%%%%%%%%%%%%%%%%%%%%%%%%%%%
\textbf{Request \#3:}
\begin{lstlisting}[xleftmargin=1pc,numbers=none]
PUT /robot/1/position HTTP/1.1
Content-Type: application/json

{
  "position": {"x": 5, "y": -2}
}
\end{lstlisting}
\textbf{Response \#3:}
\begin{lstlisting}[xleftmargin=1pc,numbers=none]
HTTP/1.1 204 No Content
\end{lstlisting}

\item  %%%%%%%%%%%%%%%%%%%%%%%%%%%%%%%%%%
\textbf{Request \#4:}
\begin{lstlisting}[xleftmargin=1pc,numbers=none]
POST /alien/abcdefghijklmnop/report HTTP/1.1
Content-Type: application/json

{
  "robot_id": 1,
  "distance": 2.236
}
\end{lstlisting}
\textbf{Response \#4:}
\begin{lstlisting}[xleftmargin=1pc,numbers=none]
HTTP/1.1 200 OK
\end{lstlisting}

\newpage
\item  %%%%%%%%%%%%%%%%%%%%%%%%%%%%%%%%%%
\textbf{Request \#5:}
\begin{lstlisting}[xleftmargin=1pc,numbers=none]
GET /alien/abcdefghijklmnop/position HTTP/1.1
\end{lstlisting}
\textbf{Response \#5:}
\begin{lstlisting}[xleftmargin=1pc,numbers=none]
HTTP/1.1 424 Failed Dependency
\end{lstlisting}

\item  %%%%%%%%%%%%%%%%%%%%%%%%%%%%%%%%%%
\textbf{Request \#6:}
\begin{lstlisting}[xleftmargin=1pc,numbers=none]
PUT /robot/2/position HTTP/1.1
Content-Type: application/json

{
  "position": {"x": 3, "y": -3}
}
\end{lstlisting}
\textbf{Response \#6:}
\begin{lstlisting}[xleftmargin=1pc,numbers=none]
HTTP/1.1 204 No Content
\end{lstlisting}

\item  %%%%%%%%%%%%%%%%%%%%%%%%%%%%%%%%%%
\textbf{Request \#7:}
\begin{lstlisting}[xleftmargin=1pc,numbers=none]
POST /alien/abcdefghijklmnop/report HTTP/1.1
Content-Type: application/json

{
  "robot_id": 2,
  "distance": 2.000
}
\end{lstlisting}
\textbf{Response \#7:}
\begin{lstlisting}[xleftmargin=1pc,numbers=none]
HTTP/1.1 200 OK
\end{lstlisting}

\item  %%%%%%%%%%%%%%%%%%%%%%%%%%%%%%%%%%
\textbf{Request \#8:}
\begin{lstlisting}[xleftmargin=1pc,numbers=none]
GET /alien/abcdefghijklmnop/position HTTP/1.1
\end{lstlisting}
\textbf{Response \#8:}
\begin{lstlisting}[xleftmargin=1pc,numbers=none]
HTTP/1.1 200 OK
Content-Type: application/json

{
  "position": {"x": 3, "y": -1}
}
\end{lstlisting}

\end{itemize}
