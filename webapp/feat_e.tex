%!TEX root = robots_desc.tex

%  _____          _                    _____      _     _____ 
% |  ___|__  __ _| |_ _   _ _ __ ___  | ____|_  _| |_  | ____|
% | |_ / _ \/ _` | __| | | | '__/ _ \ |  _| \ \/ / __| |  _|  
% |  _|  __/ (_| | |_| |_| | | |  __/ | |___ >  <| |_  | |___ 
% |_|  \___|\__,_|\__|\__,_|_|  \___| |_____/_/\_\\__| |_____|
%

\setSpacing{1.5}
\section{Feature Extension: E}

\begin{quote}
  \footnotesize
  \textbf{หมายเหตุ:} Web Service Application ของผู้เข้าแข่งขันจำเป็นต้อง Implement Feature Extension A ก่อนจึงจะ Implement Feature ใหม่ต่อไปนี้ได้
\end{quote}

\noindent
เราจะเพิ่ม Feature ให้แก่ Web Service Application ดังนี้
\begin{itemize}[topsep=0pc,itemsep=0pc]
\item 
    Operator ที่เป็นมนุษย์สามารถ Query เพื่อสอบถามว่า Robot สองตัวที่อยู่ใกล้กันมากที่สุด ณ เวลาปัจจุบันนั้นมีระยะห่างกันเท่าใด (ภายใต้ Euclidean Distance Metric)
\end{itemize}

\setSpacing{1.5}
\subsection{API Endpoints}

\noindent
สำหรับ Feature Extension นี้จะต้อง Implement HTTP Endpoint เพิ่มเติมอีก 1 จุดคือ

\begin{description}[parsep=0.5pc]
\item[\npt{GET}{/closestpair}] ~ \\
    Query เพื่อให้ Web Service ค้นหาระยะห่างระหว่าง Robot คู่ใด ๆ ที่มีค่าน้อยที่สุด
    โดยอ้างอิงข้อมูลตำแหน่งของ Robot ที่อัปเดตล่าสุด

    Response Object จะมี JSON Field เดียวคือ \lstinline{"distance"} ซึ่งระบุค่าระยะห่างที่น้อยที่สุดระหว่าง Robot คู่ใด ๆ
\end{description}

\noindent        
รายละเอียดเพิ่มเติมสามารถหาอ่านได้จากไฟล์ API Specification ที่แนบมากับเอกสารนี้

\setSpacing{1.5}
\subsection{Example Requests and Responses}

\noindent
พิจารณาลำดับของการเรียก API Endpoints ต่อไปนี้

\begin{itemize}
\item  %%%%%%%%%%%%%%%%%%%%%%%%%%%%%%%%%%
\textbf{Request \#1:}
\begin{lstlisting}[xleftmargin=1pc,numbers=none]
PUT /robot/1/position HTTP/1.1
Content-Type: application/json

{
  "position": {"x": -1, "y": 1}
}
\end{lstlisting}
\textbf{Response \#1:}
\begin{lstlisting}[xleftmargin=1pc,numbers=none]
HTTP/1.1 204 No Content
\end{lstlisting}

\item  %%%%%%%%%%%%%%%%%%%%%%%%%%%%%%%%%%
\textbf{Request \#2:}
\begin{lstlisting}[xleftmargin=1pc,numbers=none]
GET /closestpair HTTP/1.1
\end{lstlisting}
\newpage
\textbf{Response \#2:}
\begin{lstlisting}[xleftmargin=1pc,numbers=none]
HTTP/1.1 424 Failed Dependency
\end{lstlisting}

\item  %%%%%%%%%%%%%%%%%%%%%%%%%%%%%%%%%%
\textbf{Request \#3:}
\begin{lstlisting}[xleftmargin=1pc,numbers=none]
PUT /robot/2/position HTTP/1.1
Content-Type: application/json

{
  "position": {"x": 3, "y": -3}
}
\end{lstlisting}
\textbf{Response \#3:}
\begin{lstlisting}[xleftmargin=1pc,numbers=none]
HTTP/1.1 204 No Content
\end{lstlisting}

\item  %%%%%%%%%%%%%%%%%%%%%%%%%%%%%%%%%%
\textbf{Request \#4:}
\begin{lstlisting}[xleftmargin=1pc,numbers=none]
PUT /robot/4/position HTTP/1.1
Content-Type: application/json

{
  "position": {"x": 5, "y": -2}
}
\end{lstlisting}
\textbf{Response \#4:}
\begin{lstlisting}[xleftmargin=1pc,numbers=none]
HTTP/1.1 204 No Content
\end{lstlisting}

\item  %%%%%%%%%%%%%%%%%%%%%%%%%%%%%%%%%%
\textbf{Request \#5:}
\begin{lstlisting}[xleftmargin=1pc,numbers=none]
GET /closestpair HTTP/1.1
\end{lstlisting}
\textbf{Response \#5:}
\begin{lstlisting}[xleftmargin=1pc,numbers=none]
HTTP/1.1 200 OK
Content-Type: application/json

{
  "distance": 2.236
}
\end{lstlisting}

\end{itemize}
