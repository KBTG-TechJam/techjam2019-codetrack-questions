%!TEX root = robots_desc.tex

%  ___       _ _   _       _   _____          _
% |_ _|_ __ (_) |_(_) __ _| | |  ___|__  __ _| |_ _   _ _ __ ___  ___
%  | || '_ \| | __| |/ _` | | | |_ / _ \/ _` | __| | | | '__/ _ \/ __|
%  | || | | | | |_| | (_| | | |  _|  __/ (_| | |_| |_| | | |  __/\__ \
% |___|_| |_|_|\__|_|\__,_|_| |_|  \___|\__,_|\__|\__,_|_|  \___||___/
%

\setSpacing{1.5}
\section{Baseline Features}

Web Service Application ที่เขียนขึ้นจะต้องอำนวยความสะดวกให้แก่ Robot ที่ถูกปล่อยให้ไปสำรวจพื้นที่สองมิติ อันประกอบไปด้วย Feature พื้นฐานดังนี้
\begin{itemize}[topsep=0pc,itemsep=0pc]
\item  
    Robot แต่ละตัวสามารถ Query เพื่อสอบถามระยะห่างระหว่างจุด 2 จุดกำหนดให้ได้
\item
    ข้อมูลตำแหน่งของจุดพิกัด $(x, y) \in \mathbb{Z}^2$ ที่จะปรากฏใน Input Data หรือ Output Data จะสามารถเขียนในรูปของ Object ที่มี 2 Subfield ซึ่งได้แก่ \lstinline{"x"} และ \lstinline{"y"}
\end{itemize}

\setSpacing{1.5}
\subsection{API Endpoints}

\begin{description}[parsep=0.5pc]
\item[\npt{POST}{/distance}] ~ \\
    Query เพื่อคำนวณระยะห่างระหว่างจุด 2 จุดบนระนาบสองมิติ
    
    โดย Request Object จะประกอบไปด้วย JSON Field \lstinline{"first_pos"} และ \lstinline{"second_pos"} ซึ่งระบุถึงตำแหน่งของจุดทั้งสองจุดของ Input Data \;
    จุดแต่ละจุดของ Input Data จะอยู่ในรูปของพิกัดจำนวนเต็ม $(x, y) \in \mathbb{Z}^2$
    
    ส่วน Response Object จะประกอบไปด้วย JSON Field เดียวคือ \lstinline{"distance"} ซึ่งเป็น Euclidean Distance ระหว่างจุดสองจุดดังกล่าว โดยคำตอบจะต้องคาดเคลื่อนไม่เกิน 0.001 จากคำตอบจริง
\end{description}

\noindent        
รายละเอียดเพิ่มเติมสามารถหาอ่านได้จากไฟล์ API Specification ที่แนบมากับเอกสารนี้

\setSpacing{1.5}
\subsection{Example Requests and Responses}

\noindent
พิจารณาลำดับของการเรียก API Endpoints ต่อไปนี้

\begin{itemize}
\item  %%%%%%%%%%%%%%%%%%%%%%%%%%%%%%%%%%
\textbf{Request \#1:}
\begin{lstlisting}[xleftmargin=1pc,numbers=none]
POST /distance HTTP/1.1
Content-Type: application/json

{
  "first_pos": {"x": 3, "y": -2},
  "second_pos": {"x": -1, "y": 1}
}
\end{lstlisting}
\textbf{Response \#1:}
\begin{lstlisting}[xleftmargin=1pc,numbers=none]
HTTP/1.1 200 OK
Content-Type: application/json

{
  "distance": 5
}
\end{lstlisting}

\newpage
\item  %%%%%%%%%%%%%%%%%%%%%%%%%%%%%%%%%%
\textbf{Request \#2:}
\begin{lstlisting}[xleftmargin=1pc,numbers=none]
POST /distance HTTP/1.1
Content-Type: application/json

{
  "first_pos": {"x": 0, "y": 0},
  "second_pos": {"x": 1, "y": 2}
}
\end{lstlisting}
\textbf{Response \#2:}
\begin{lstlisting}[xleftmargin=1pc,numbers=none]
HTTP/1.1 200 OK
Content-Type: application/json

{
  "distance": 2.236
}
\end{lstlisting}
\end{itemize}
