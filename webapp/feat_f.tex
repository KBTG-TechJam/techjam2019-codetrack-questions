%!TEX root = robots_desc.tex

%  _____          _                    _____      _     _____ 
% |  ___|__  __ _| |_ _   _ _ __ ___  | ____|_  _| |_  |  ___|
% | |_ / _ \/ _` | __| | | | '__/ _ \ |  _| \ \/ / __| | |_   
% |  _|  __/ (_| | |_| |_| | | |  __/ | |___ >  <| |_  |  _|  
% |_|  \___|\__,_|\__|\__,_|_|  \___| |_____/_/\_\\__| |_|    
%                                                            

\setSpacing{1.5}
\section{Feature Extension: F}

\begin{quote}
  \footnotesize
  \textbf{หมายเหตุ:} Web Service Application ของผู้เข้าแข่งขันจำเป็นต้อง Implement Feature Extension C ก่อนจึงจะ Implement Feature ใหม่ต่อไปนี้ได้
\end{quote}

\noindent
เราจะเพิ่ม Feature ให้แก่ Web Service Application ดังนี้
\begin{itemize}[topsep=0pc,itemsep=0pc]
\item 
    ใน Feature Extension C เดิมทีนั้น Endpoint \npt{POST}{/nearest} จะสามารถ Query เพื่อค้นหา Robot ที่อยู่ใกล้กับจุดที่กำหนดให้มากที่สุดเพียง 1 ตัว \;
    ในส่วนนี้ เราจะรับ Input Parameter \lstinline{k} เพิ่มเติม เพื่อระบุว่าเราต้องการค้นหา Robot ที่อยู่ใกล้กับจุดที่กำหนดให้มากที่สุดทั้งสิ้น \lstinline{k} ตัว
\end{itemize}

\setSpacing{1.5}
\subsection{Modified Changes}

\begin{itemize}[parsep=0.5pc]
\item 
    ผู้เข้าแข่งขันจะต้องแก้ไข Endpoint \npt{POST}{/nearest} เพื่อให้รองรับ JSON Field ใหม่ชื่อว่า \lstinline{"k"} ซึ่งจะเป็นจำนวนเต็มที่ระบุจำนวน Robot ที่อยู่ใกล้จุด \lstinline{"ref_position"} มากที่สุดที่ต้องการค้นหา

    Response Object จะมีอาร์เรย์ \lstinline{"robot_ids"} ความยาวไม่เกิน \lstinline{k} ซึ่งระบุ Robot ID เรียงตามลำดับระยะห่างจากจุดที่กำหนดให้ จากน้อยที่สุดไปหามากที่สุด \;
    หากมี Robot หลายตัวที่มีระยะห่างจากจุดดังกล่าวเท่านั้น ให้พิจารณา Robot ที่มีค่า ID น้อยกว่าก่อนเป็นอันดับแรก
\end{itemize}

\noindent
รายละเอียดเพิ่มเติมสามารถหาอ่านได้จากไฟล์ API Specification ที่แนบมากับเอกสารนี้

\setSpacing{1.5}
\subsection{Example Requests and Responses}

\noindent
พิจารณาลำดับของการเรียก API Endpoints ต่อไปนี้

\begin{itemize}
\item  %%%%%%%%%%%%%%%%%%%%%%%%%%%%%%%%%%
\textbf{Request \#1:}
\begin{lstlisting}[xleftmargin=1pc,numbers=none]
PUT /robot/1/position HTTP/1.1
Content-Type: application/json

{
  "position": {"x": 1, "y": 1}
}
\end{lstlisting}
\textbf{Response \#1:}
\begin{lstlisting}[xleftmargin=1pc,numbers=none]
HTTP/1.1 204 No Content
\end{lstlisting}

\item  %%%%%%%%%%%%%%%%%%%%%%%%%%%%%%%%%%
\textbf{Request \#2:}
\begin{lstlisting}[xleftmargin=1pc,numbers=none]
POST /nearest HTTP/1.1
Content-Type: application/json

{
  "ref_position": {"x": -1, "y": 1},
  "k": 2
}
\end{lstlisting}
\newpage
\textbf{Response \#2:}
\begin{lstlisting}[xleftmargin=1pc,numbers=none]
HTTP/1.1 200 OK
Content-Type: application/json

{
  "robot_ids": [1]
}
\end{lstlisting}

\item  %%%%%%%%%%%%%%%%%%%%%%%%%%%%%%%%%%
\textbf{Request \#3:}
\begin{lstlisting}[xleftmargin=1pc,numbers=none]
PUT /robot/2/position HTTP/1.1
Content-Type: application/json

{
  "position": {"x": 0, "y": 0}
}
\end{lstlisting}
\textbf{Response \#3:}
\begin{lstlisting}[xleftmargin=1pc,numbers=none]
HTTP/1.1 204 No Content
\end{lstlisting}

\item  %%%%%%%%%%%%%%%%%%%%%%%%%%%%%%%%%%
\textbf{Request \#4:}
\begin{lstlisting}[xleftmargin=1pc,numbers=none]
POST /nearest HTTP/1.1
Content-Type: application/json

{
  "ref_position": {"x": -1, "y": 1}
}
\end{lstlisting}
\textbf{Response \#4:}
\begin{lstlisting}[xleftmargin=1pc,numbers=none]
HTTP/1.1 200 OK
Content-Type: application/json

{
  "robot_ids": [2]
}
\end{lstlisting}

\item  %%%%%%%%%%%%%%%%%%%%%%%%%%%%%%%%%%
\textbf{Request \#5}
\begin{lstlisting}[xleftmargin=1pc,numbers=none]
POST /nearest HTTP/1.1
Content-Type: application/json

{
  "ref_position": {"x": -1, "y": 1},
  "k": 2
}
\end{lstlisting}
\textbf{Response \#5:}
\begin{lstlisting}[xleftmargin=1pc,numbers=none]
HTTP/1.1 200 OK
Content-Type: application/json

{
  "robot_ids": [2, 1]
}
\end{lstlisting}

\newpage
\item  %%%%%%%%%%%%%%%%%%%%%%%%%%%%%%%%%%
\textbf{Request \#6:}
\begin{lstlisting}[xleftmargin=1pc,numbers=none]
POST /nearest HTTP/1.1
Content-Type: application/json

{
  "ref_position": {"x": 1, "y": 0},
  "k": 2
  }
\end{lstlisting}
\textbf{Response \#6:}
\begin{lstlisting}[xleftmargin=1pc,numbers=none]
HTTP/1.1 200 OK
Content-Type: application/json

{
  "robot_ids": [1, 2]
}
\end{lstlisting}
\end{itemize}
