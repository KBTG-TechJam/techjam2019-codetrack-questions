%!TEX root = robots_desc.tex

%  _____          _                    _____      _      ____ 
% |  ___|__  __ _| |_ _   _ _ __ ___  | ____|_  _| |_   / ___|
% | |_ / _ \/ _` | __| | | | '__/ _ \ |  _| \ \/ / __| | |  _ 
% |  _|  __/ (_| | |_| |_| | | |  __/ | |___ >  <| |_  | |_| |
% |_|  \___|\__,_|\__|\__,_|_|  \___| |_____/_/\_\\__|  \____|
%  

\setSpacing{1.5}
\section{Feature Extension: G}

\begin{quote}
  \footnotesize
  \textbf{หมายเหตุ:} Web Service Application ของผู้เข้าแข่งขันจำเป็นต้อง Implement Feature Extension A ก่อนจึงจะ Implement Feature ใหม่ต่อไปนี้ได้
\end{quote}

\noindent
เราจะเพิ่ม Feature ให้แก่ Web Service Application ดังนี้
\begin{itemize}[topsep=0pc,itemsep=0pc,parsep=0.5pc]
\item 
    เนื่องจาก Robot บางตัวเป็นรุ่นเก่า ซึ่งทำงานตาม Legacy API Specification เราจึงจำเป็นต้องแก้ไข Endpoint 2 อัน (ได้แก่ \npt{POST}{/distance} และ \npt{PUT}{/robot/\{robot\_id\}/position})  ให้รองรับค่าพิกัด $(x,y) \in \mathbb{Z}^2$ ที่เป็น \hrsp“Legacy Format X”{\hrsp} ได้อีกด้วย

    กล่าวคือจากเดิมที่ค่าพิกัดจะระบุด้วย Object ที่ประกอบไปด้วย 2 Subfield \lstinline{"x"} และ \lstinline{"y"} \; 
    ใน Legacy Format X นี้จะใช้ Subfield \lstinline{"east"} หรือ \lstinline{"west"} แทนค่าพิกัด \lstinline{"x"} เดิม (ในทิศบวกและลบตามลำดับ) \; 
    และจะใช้ Subfield \lstinline{"north"} ละ \lstinline{"south"} แทนค่าพิกัด \lstinline{"y"} เดิม (ในทิศบวกและลบตามลำดับ)

    ยกตัวอย่างเช่น \lstinline'{"x": 0, "y": -1}' สามารถเขียนเป็น \lstinline'{"south": 1, "east": 0}' แทนได้ หรือ \lstinline'{"x": -2, "y": 3}' สามารถเขียนเป็น \lstinline'{"north": 3, "west": 2}' แทนได้ เป็นต้น
\end{itemize}

\setSpacing{1.5}
\subsection{Modified Changes}

\begin{itemize}[parsep=0.5pc]
\item 
    ผู้เข้าแข่งขันจะต้องแก้ไข Endpoint \npt{POST}{/distance} เพื่อให้ JSON Field \lstinline{"first_pos"} และ \\\lstinline{"second_pos"} ของ Request Object สามารถรองรับพิกัดแบบ Legacy Format X ที่ระบุข้างต้นได้
\item 
    ผู้เข้าแข่งขันจะต้องแก้ไข Endpoint \npt{PUT}{/robot/\{robot\_id\}/position} เพื่อให้ JSON Field \\\lstinline{"position"} ของ Request Object สามารถรองรับพิกัดแบบ Legacy Format X ที่ระบุข้างต้นได้
\end{itemize}

\noindent
รายละเอียดเพิ่มเติมสามารถหาอ่านได้จากไฟล์ API Specification ที่แนบมากับเอกสารนี้

\setSpacing{1.5}
\subsection{Example Requests and Responses}

\noindent
พิจารณาลำดับของการเรียก API Endpoints ต่อไปนี้

\begin{itemize}
\item  %%%%%%%%%%%%%%%%%%%%%%%%%%%%%%%%%%
\textbf{Request \#1:}
\begin{lstlisting}[xleftmargin=1pc,numbers=none]
POST /distance HTTP/1.1
Content-Type: application/json

{
  "first_pos": {"south": 2, "east": 3},
  "second_pos": {"north": 1, "west": 1}
}
\end{lstlisting}
\newpage
\textbf{Response \#1:}
\begin{lstlisting}[xleftmargin=1pc,numbers=none]
HTTP/1.1 200 OK
Content-Type: application/json

{
  "distance": 5
}
\end{lstlisting}

\item  %%%%%%%%%%%%%%%%%%%%%%%%%%%%%%%%%%
\textbf{Request \#2:}
\begin{lstlisting}[xleftmargin=1pc,numbers=none]
PUT /robot/1/position HTTP/1.1
Content-Type: application/json

{
  "position": {"east": 3, "south": 4}
}
\end{lstlisting}
\textbf{Response \#2:}
\begin{lstlisting}[xleftmargin=1pc,numbers=none]
HTTP/1.1 204 No Content
\end{lstlisting}

\end{itemize}
