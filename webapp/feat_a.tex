%!TEX root = robots_desc.tex

%  _____          _                    _____      _        _
% |  ___|__  __ _| |_ _   _ _ __ ___  | ____|_  _| |_     / \
% | |_ / _ \/ _` | __| | | | '__/ _ \ |  _| \ \/ / __|   / _ \
% |  _|  __/ (_| | |_| |_| | | |  __/ | |___ >  <| |_   / ___ \
% |_|  \___|\__,_|\__|\__,_|_|  \___| |_____/_/\_\\__| /_/   \_\
%

\setSpacing{1.5}
\section{Feature Extension: A}

\noindent
เราจะเพิ่ม Feature ให้แก่ Web Service Application ดังนี้
\begin{itemize}[topsep=0pc,itemsep=0pc]
\item  
    Robot แต่ละตัวสามารถส่งข้อมูลตำแหน่งปัจจุบันของตนเองมาอัปเดตให้กับ Web Service กลางได้ \\
    (Robot แต่ละตัวจะระบุตัวเองด้วย ID ซึ่งเป็นจำนวนเต็มในช่วง $1$ ถึง $999,\!999$)
\item 
    ใครก็ได้สามารถ Query สอบถามตำแหน่งที่ทราบล่าสุดของ Robot ตัวหนึ่งที่กำหนดให้ได้
\item 
    เดิมทีนั้นเราจะอ้างอิงถึงจุดบนระนาบสองมิติด้วยระบบพิกัด $(x, y)$ แต่สำหรับ Feature นี้เราจะซัพพอร์ตวิธีการอ้างอิงถึงจุดบนระนาบสองมิติด้วย Robot ID \;
    กล่าวคือเราจะใช้ ID ของ Robot เพื่อระบุถึงตำแหน่งพิกัดล่าสุดที่ Robot ดังกล่าวรายงานอัปเดตเข้ามาในระบบ Service กลาง
    (ผ่าน Feature ใหม่ที่ระบุในข้อแรก)
\end{itemize}

\setSpacing{1.5}
\subsection{API Endpoints}

\noindent
สำหรับ Feature Extension นี้จะต้อง Implement HTTP Endpoint เพิ่มเติมอีก 2 จุดคือ

\begin{description}[parsep=0.5pc]
\item[\npt{PUT}{/robot/\{robot\_id\}/position}] ~ \\
    Update ข้อมูลตำแหน่งปัจจุบันของ Robot ที่กำหนดให้ \;
    โดย Path Parameter \lstinline{robot_id} จะระบุ Robot ID ของ Robot ดังกล่าว และข้อมูลจาก JSON Field \lstinline{"position"} จะระบุตำแหน่งปัจจุบันของ Robot ดังกล่าว \;
    (โดยให้นำข้อมูลนี้ไปแทนที่ข้อมูลเก่าของ Robot ข้างต้นหากมีข้อมูลเก่าอยู่แล้ว)
\item[\npt{GET}{/robot/\{robot\_id\}/position}] ~ \\
    Query เพื่อเรียกดูข้อมูลตำแหน่งที่อัปเดตล่าสุดของ Robot ที่กำหนดให้ \;
    โดย Path Parameter \lstinline{robot_id} จะระบุ Robot ID ของ Robot
\end{description}

\noindent        
รายละเอียดเพิ่มเติมสามารถหาอ่านได้จากไฟล์ API Specification ที่แนบมากับเอกสารนี้

\setSpacing{1.5}
\subsection{Modified Changes}

\begin{itemize}[parsep=0.5pc]
\item
    ดังที่ได้เกริ่นไว้ข้างต้นแล้ว ผู้เข้าแข่งขันจะต้องแก้ไข Endpoint \npt{POST}{/distance} เพื่อให้ JSON Field \lstinline{"first_pos"} และ \lstinline{"second_pos"} สามารถรองรับการระบุตำแหน่งบนพื้นที่สองมิติด้วย Robot ID ได้ โดยข้อมูล Robot ID จะมาในรูปแบบของสตริงที่สอดคล้องกับ Regular Expression ดังต่อไปนี้
    \begin{center}
        \lstinline{"^robot#([1-9][0-9]*)$"}
    \end{center}
    และตำแหน่งพิกัดของจุดดังกล่าวจะเป็นค่าตำแหน่งของ Robot ข้างต้นนี้ที่ถูกอัปเดตล่าสุดผ่าน Endpoint \npt{PUT}{/robot/\{robot\_id\}/position}
\end{itemize}

\newpage
\setSpacing{1.5}
\subsection{Example Requests and Responses}

\noindent
พิจารณาลำดับของการเรียก API Endpoints ต่อไปนี้

\begin{itemize}
\item  %%%%%%%%%%%%%%%%%%%%%%%%%%%%%%%%%%
\textbf{Request \#1:}
\begin{lstlisting}[xleftmargin=1pc,numbers=none]
PUT /robot/1/position HTTP/1.1
Content-Type: application/json

{
  "position": {"x": 3, "y": 4}
}
\end{lstlisting}
\textbf{Response \#1:}
\begin{lstlisting}[xleftmargin=1pc,numbers=none]
HTTP/1.1 204 No Content
\end{lstlisting}

\item  %%%%%%%%%%%%%%%%%%%%%%%%%%%%%%%%%%
\textbf{Request \#2:}
\begin{lstlisting}[xleftmargin=1pc,numbers=none]
POST /distance HTTP/1.1
Content-Type: application/json

{
  "first_pos": "robot#1",
  "second_pos": {"x": 0, "y": 8}
}
\end{lstlisting}
\textbf{Response \#2:}
\begin{lstlisting}[xleftmargin=1pc,numbers=none]
HTTP/1.1 200 OK
Content-Type: application/json

{
  "distance": 5
}
\end{lstlisting}

\item  %%%%%%%%%%%%%%%%%%%%%%%%%%%%%%%%%%
\textbf{Request \#3:}
\begin{lstlisting}[xleftmargin=1pc,numbers=none]
GET /robot/1/position HTTP/1.1
\end{lstlisting}
\textbf{Response \#3:}
\begin{lstlisting}[xleftmargin=1pc,numbers=none]
HTTP/1.1 200 OK
Content-Type: application/json

{
  "position": {"x": 3, "y": 4}
}
\end{lstlisting}

\newpage
\item  %%%%%%%%%%%%%%%%%%%%%%%%%%%%%%%%%%
\textbf{Request \#4:}
\begin{lstlisting}[xleftmargin=1pc,numbers=none]
GET /robot/2/position HTTP/1.1
\end{lstlisting}
\textbf{Response \#4:}
\begin{lstlisting}[xleftmargin=1pc,numbers=none]
HTTP/1.1 404 Not Found
\end{lstlisting}

\item  %%%%%%%%%%%%%%%%%%%%%%%%%%%%%%%%%%
\textbf{Request \#5:}
\begin{lstlisting}[xleftmargin=1pc,numbers=none]
PUT /robot/1/position HTTP/1.1
Content-Type: application/json

{
  "position": {"x": 0, "y": 0}
}
\end{lstlisting}
\textbf{Response \#5:}
\begin{lstlisting}[xleftmargin=1pc,numbers=none]
HTTP/1.1 204 No Content
\end{lstlisting}

\item  %%%%%%%%%%%%%%%%%%%%%%%%%%%%%%%%%%
\textbf{Request \#6:}
\begin{lstlisting}[xleftmargin=1pc,numbers=none]
PUT /robot/2/position HTTP/1.1
Content-Type: application/json

{
  "position": {"x": 1, "y": 1}
}
\end{lstlisting}
\textbf{Response \#6:}
\begin{lstlisting}[xleftmargin=1pc,numbers=none]
HTTP/1.1 204 No Content
\end{lstlisting}

\item  %%%%%%%%%%%%%%%%%%%%%%%%%%%%%%%%%%
\textbf{Request \#7:}
\begin{lstlisting}[xleftmargin=1pc,numbers=none]
POST /distance HTTP/1.1
Content-Type: application/json

{
  "first_pos": "robot#2",
  "second_pos": "robot#1"
}
\end{lstlisting}
\textbf{Response \#7:}
\begin{lstlisting}[xleftmargin=1pc,numbers=none]
HTTP/1.1 200 OK
Content-Type: application/json

{
  "distance": 1.414
}
\end{lstlisting}

\end{itemize}
