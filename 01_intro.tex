%!TEX root = book.tex
\chapter{เกี่ยวกับการแข่งขัน}

กสิกร\:บิซิเนส{\hrsp}–{\hrsp}เทคโนโลยี\:กรุ๊ป {\ltspc (KBTG)} ได้จัดการแข่งขัน \textbf{TechJam} ขึ้นทุกปีเพื่อค้นหาสุดยอดผู้เชี่ยวชาญทางด้านเทคโนโลยี ทั้งในด้าน Code\,·\,Data\,·\,Design ที่จะมาผสานกันเป็นเครื่องจักรกำลังสูงที่จะขับเคลื่อนวงการเทคโนโลยีในประเทศไทยไปข้างหน้าต่อไป

สำหรับ \textbf{TechJam\:Code\:Track} นั้น เรามุ่งเน้นที่จะค้นหาโปรแกรมเมอร์ที่มีทักษะและความสามารถรอบด้าน ทั้งในเรื่องของการคิดวิเคราะห์\,แก้ปัญหาเชิงคำนวณ ในเรื่องความรู้ความสามารถเกี่ยวกับคอมพิวเตอร์\,โครงสร้างข้อมูล\,และอัลกอริทึม และในเรื่องของทักษะในการพัฒนาซอฟต์แวร์ที่ดีและได้มาตรฐาน\;\;
ซอฟต์แวร์ที่ดีนั้นสามารถมองได้หลายปัจจัย เช่น

\begin{itemize}
\item  ซอฟต์ทำงานได้อย่างถูกต้อง สามารถประมวลผลข้อมูลได้แม่นยำ\,และทำงานได้สอดคล้องกับสเปกที่วางไว้
\item  ซอฟต์แวร์ทำงานได้อย่างมีประสิทธิภาพ เช่น ประมวลผลได้เร็ว ใช้ทรัพยากรได้อย่างคุ้มค่า เป็นต้น
\item  ซอฟต์แวร์มีการวางโครงสร้างที่ดี มีความยืดหยุ่น ทำให้สามารถต่อเติมหรือแก้ไขโค้ดได้ง่ายเมื่อต้องการ และบำรุงดูแลรักษาได้ง่ายในระยะยาว
\end{itemize} 

การแข่งขัน \textbf{TechJam\:Code\:Track} ในปีก่อน\,ๆ จะเน้นการแก้ปัญหาเชิงอัลกอริทึมเป็นหลัก\;\;
แต่สำหรับการแข่งขัน \textbf{TechJam\,2019\:DEEP\:CODE} ในปีนี้ เราปรับเปลี่ยนรูปแบบการแข่งขันเพื่อทดสอบความสามารถของโปรแกรมเมอร์ในมุมมองที่กว้างขึ้น\;\;
กล่าวคือเราได้สร้างการแข่งขันรูปแบบใหม่ นั่นก็คือการพัฒนาแอปพลิเคชันตามโจทย์ที่กำหนดให้ภายใต้สถานการณ์จำลองเสมือนจริง เพื่อทดสอบความสามารถของผู้เข้าแข่งขันในการรับมือโจทย์ที่มีการเปลี่ยนแปลง Requirement หรือ Specification ที่เกิดขึ้นหลายครั้งระหว่างการแข่งขัน

อย่างไรก็ตาม {\ltspc KBTG} มิได้คาดหวังที่จะค้นหาสุดยอดโปรแกรมเมอร์เพียงอย่างเดียว\;\;
เราหวังว่าการแข่งขัน \textbf{TechJam} จะจุดประกายความสนใจทางด้านเทคโนโลยีให้แก่ชุมชนนักพัฒนาซอฟต์แวร์และคนรุ่นใหม่ที่จะมาเป็นกำลังสำคัญในวงการเทคโนโลยีต่อไป

\vspace{2pc}
\hfill\rule[2.5pt]{2em}{0.75pt} ทีมงาน \textbf{TechJam\,2019\:DEEP\:CODE}

\setSpacing{1.5}
\subsection*{ลักษณะของโจทย์}

\noindent
โจทย์ที่ใช้ในการแข่งขัน \textbf{TechJam\,2019\:DEEP\:CODE} มี 3 รูปแบบดังนี้ 

\begin{description}
\item[\textthai{โจทย์ปัญหาเชิงอัลกอริทึม}] \phantom{anchor} \\
    ผู้เข้าแข่งขันจะต้องเขียนโปรแกรมเพื่อรับข้อมูลนำเข้า ({\hrsp}Input Data{\hrsp}) ไปประมวลผลและคืนออกมาเป็นข้อมูลส่งออก ({\hrsp}Output Data{\hrsp})\;\;
    โจทย์จะระบุว่าข้อมูลนำเข้ามีลักษณะอย่างไรและต้องการให้นำไปประมวลผลออกมาเป็นข้อมูลส่งออกในลักษณะใด\;\;
    ผู้เข้าแข่งขันจะต้องคิดวิเคราะห์สารจากโจทย์แล้วจึงเลือกสรรโครงสร้างข้อมูลและอัลกอริทึมที่เหมาะสมกับงานมาเขียนโปรแกรมให้ทำงานตามที่โจทย์ระบุ
    
\item[\textthai{โจทย์ปริศนาลับสมอง}] \phantom{anchor} \\
    ผู้เข้าแข่งขันจะต้องไขปริศนาหาคำตอบที่ถูกต้อง โดยจะใช้เทคนิคและเครื่องมือใดในการช่วยค้นหาคำตอบ

\item[\textthai{โจทย์เขียน Containerized Application}] \phantom{anchor} \\
    ผู้เข้าแข่งขันจะต้องเขียน Web Service Application ตาม Requirement และ Specification ที่กำหนดให้\;\;
    แต่โจทย์จะสามารถเปลี่ยนแปลงและถูกแก้ไขได้ตลอดระยะเวลาในการแข่งขัน\;\;ผู้เข้าแข่งขันไม่ทราบว่าล่วงหน้าว่าจะมี Requirement ส่วนใดเปลี่ยนบ้าง\:และจะมีการประกาศเปลี่ยนแปลงเมื่อใด
\end{description}


\medskip
\setSpacing{1.5}
\subsection*{รูปแบบการแข่งขัน}

\noindent
การแข่งขัน \textbf{TechJam\,2019\:DEEP\:CODE} แบ่งออกเป็น 3 รอบหลัก ๆ ได้แก่

\begin{description}
\item[\textthai{บททดสอบคัดกรองรอบลงทะเบียน}] \phantom{anchor} \\
    ผู้เข้าแข่งขันจะต้องเขียนโปรแกรมแก้ปัญหาเชิงอัลกอริทึมอย่างง่ายจำนวน 3 ข้อผ่านระบบตรวจออนไลน์\;\;
    ผู้เข้าแข่งขันที่ทำคะแนนรวมได้ 50\% ขึ้นไปจะทีสิทธิแข่งขันใน First Round ต่อไป

\item[First Round] \phantom{anchor} \\
    ผู้เข้าแข่งขันจะต้องเขียนโปรแกรมแก้ปัญหาเชิงอัลกอริทึมแสนท้าทายจำนวน 3 ข้อผ่านระบบตรวจออนไลน์\;\;
    ผู้เข้าแข่งขันที่ทำคะแนนรวมได้ดีที่สุดประมาณ 30 ทีมจะได้รับเชิญให้มาแข่งใน Final Round 

\item[Final Round] \phantom{anchor} \\
    เป็นการแข่งขันเขียนโปรแกรมเต็มรูปแบบที่อาคาร {\ltspc KBTG} แบ่งออกเป็นการแข่งขันภาคเช้าและภาคบ่าย\;\;
    ในช่วงเช้าจะเป็นการแก้ปัญหาเชิงอัลกอริทึมผสมกับการไขโจทย์ปริศนาลับสมอง\:ภายในเวลา 3 ชั่วโมง\;\;
    ส่วนในช่วงบ่ายจะเป็นการเขียนซอฟต์แวร์เป็น Containerized Web Service Application ตาม Requirement ที่กำหนดภายในเวลา 4 ชั่วโมง\;\;
    คะแนนของการแข่งขันแต่ละส่วนถูกแบ่งออกเป็นดังนี้
    \begin{itemize}[topsep=0.5pc]
        \item{} [90 คะแนน]\: \textbf{โจทย์ปัญหาเชิงอัลกอริทึม}\: 3 ข้อ ข้อละ 30 คะแนน
        \item{} [30 คะแนน]\: \textbf{โจทย์ปริศนาลับสมอง}\: 5 ข้อ ข้อละ 6 คะแนน
        \item{} [80 คะแนน]\: \textbf{โจทย์เขียน Containerized Application}\: 1 ตัว
    \end{itemize}
\end{description}
