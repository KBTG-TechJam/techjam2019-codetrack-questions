%!TEX root = book.tex
\section{Irregular Sequence}

\setSpacing{1.5}
\subsection{Problem Statement}

กำหนดให้ \lstinline{S} เป็น Sequence ของจำนวนเต็ม \lstinline{S[0]}, \lstinline{S[1]}, \ldots, \lstinline{S[n-1]} ซึ่งมีค่าอยู่ในช่วง \lstinline{1} ถึง \lstinline{k} \;
Sequence \lstinline{S} จะถือว่าเป็น \textbf{Irregular Sequence} ก็ต่อเมื่อใน Sequence ดังกล่าวไม่ปรากฏรูปแบบที่มีความซ้ำซ้อนบางประการ กล่าวคือ Sequence \lstinline{S} ดังกล่าวจะมีสมบัติ 2 ข้อดังต่อไปนี้ จึงจะถือว่าเป็น Irregular Sequence\;

\begin{enumerate}
    \item 
        ไม่มีจำนวนที่ซ้ำกันอยู่ติดกัน (หมายถึง \lstinline{S[i-1] ≠ S[i]} สำหรับ \lstinline{i} \lstinline{=} \lstinline{1}, \lstinline{2}, \ldots, \lstinline{n-1})
    \item
        ไม่มี Subsequence (ลำดับย่อย) ความยาว 4 จำนวนชุดใด ๆ ที่เป็น Alternating Subsequence (หรือกล่าวอีกนัยหนึ่งคือสำหรับ \lstinline{w}, \lstinline{x}, \lstinline{y}, \lstinline{z} ใด ๆ ซึ่ง \lstinline{0 ≤ w < x < y < z ≤ n-1} จะพบว่า \lstinline{S[w] = S[y] ≠ S[x] = S[z]} ไม่เป็นความจริง) 
\end{enumerate}

\bigskip
ลองพิจารณาตัวอย่างต่อไปนี้โดยสมมติว่า \lstinline{k = 5}

\begin{itemize}[leftmargin=4pc]
    \item  Sequence \lstinline{[1, 2, 3, 4, 5]} จะเป็น Irregular Sequence
    \item  Sequence \lstinline{[1, 2, 3, 1, 4]} จะเป็น Irregular Sequence เช่นกัน
    \item  Sequence \lstinline{[1, 2, 3, 3, 4]} จะไม่เป็น Irregular Sequence \\
        เพราะปรากฏเลข \lstinline{3} ติดกันใน Sequence ดังกล่าว
    \item  Sequence \lstinline{[1, 2, 3, 1, 3]} จะไม่เป็น Irregular Sequence \\ 
        เพราะมี Alternating Subsequence \lstinline{[1, 3, 1, 3]} ปรากฏอยู่
\end{itemize}

สำหรับโจทย์ข้อนี้ จะมี Input Data เป็น Irregular Sequence \lstinline{P} ซึ่งมีขนาดเล็ก \;
เราต้องการเติมจำนวนเต็มจากช่วง \lstinline{1} ถึง \lstinline{k} ต่อท้าย Sequence \lstinline{P} นี้ให้ยาวที่สุดโดยที่ Sequence ผลลัพธ์ยังคงเป็น Irregular Sequence อยู่เช่นเดิม \;
กล่าวอีกนัยหนึ่งคือ เราต้องการหา Irregular Sequence \lstinline{S} ที่ยาวที่สุดที่มี Sequence \lstinline{P} เป็น Prefix \;
หากมีคำตอบ \lstinline{S} หลายคำตอบ ให้ตอบ Sequence อันแรกสุดตามลำดับของ Lexicographical Ordering


\setSpacing{1.5}
\subsubsection{Lexicographical Ordering}

\noindent
กำหนดให้ $S_1$ และ $S_2$ คือ Sequence สองอันที่มีความยาวเท่ากัน \;
แล้ว $S_1$ จะมาก่อน $S_2$ ใน Lexicographical Ordering ก็ต่อเมื่อมีจำนวนเต็ม $j$ ซึ่ง $S_1[\,i\,] = S_2[\,i\,]$ สำหรับทุก ๆ $0 \leq i < j$ และ $S_1[\,j\,] < S_2[\,j\,]$


\setSpacing{1.5}
\subsection{Situation Example}

\noindent
สมมติว่า \lstinline{k = 5} และ Input Sequence \lstinline{P} มีค่าดังต่อไปนี้

\begin{lstlisting}[numbers=none]
P = [1, 3, 5, 2, 3]
\end{lstlisting}

\noindent
แล้วจะพบว่ามี Irregular Sequence \lstinline{S} ที่เป็นส่วนต่อขยายของ Input Sequence \lstinline{P} ได้สองรูปแบบคือ 

\begin{lstlisting}[numbers=none]
S = [1, 3, 5, 2, 3, 4, 3, 1]  # Possibility #1
S = [1, 3, 5, 2, 3, 1, 4, 1]  # Possibility #2
\end{lstlisting}

\noindent
ในบรรดารคำตอบที่เป็นไปได้ข้างต้น โปรแกรมควรให้คำตอบที่สอง \lstinline{S = [1, 3, 5, 2, 3, 1, 4, 1]} เนื่อง จากเป็นคำตอบที่ปรากฏแรกสุดในลำดับ Lexicographical Ordering


\setSpacing{1.5}
\subsection{Objectives}

\noindent
จงเขียนโปรแกรมเพื่อรับ Input Data ต่อไปนี้

\begin{itemize}[topsep=0pc,itemsep=0pt]
    \item 
        จำนวนเต็ม \lstinline{k}\: (โดยที่ \lstinline{2 ≤ k ≤ 200,000})\;ระบุช่วงจำนวนที่สามารถปรากฏใน Sequence ได้
    \item 
        Sequence \lstinline{P}\:
        (ประกอบด้วยจำนวนเต็ม \lstinline{P[0]}, \lstinline{P[1]}, \ldots, \lstinline{P[m-1]}\:
        โดยที่ \lstinline{1 ≤ m ≤ 200,000}\; 
        และสมาชิก \lstinline{P[i]} แต่ละจำนวนจะมีค่าสอดคล้องกับเงื่อนไข \lstinline{1 ≤ P[i] ≤ k})
\end{itemize}

\noindent
แล้วจึงคำนวณ \uline{Irregular Sequence \lstinline{S} ที่ยาวที่สุด} ซึ่ง 

\begin{itemize}[topsep=0pc,itemsep=0pt]
    \item 
        ประกอบไปด้วยจำนวนเต็มในช่วง \lstinline{1} ถึง \lstinline{k}
    \item
        มี Input Sequence \lstinline{P} เป็น Prefix 
    \item 
        และเป็นคำตอบที่พบแรกสุดเพื่อพิจารณาตามลำดับ Lexicographical Ordering
\end{itemize}

\noindent
และคืนค่า Sequence \lstinline{S} ดังกล่าวเป็น Output Data ของโปรแกรม


\setSpacing{1.5}
\subsection{Interfaces and Data Format}

\noindent
โปรแกรมที่เขียนขึ้นจะต้องรับ Input Data ผ่าน Standard Input ซึ่งมีรูปแบบดังต่อไปนี้

\begin{itemize}
    \item
        บรรทัดแรกมีจำนวนเต็มสองจำนวน \lstinline{k} และ \lstinline{m}
    \item 
        บรรทัดที่ \lstinline{i+2} 
        สำหรับ \lstinline{i} \lstinline{=} \lstinline{0}, \lstinline{1}, \ldots, \lstinline{m-1}
        จะมีจำนวนเต็มหนึ่งจำนวน ซึ่งก็คือ \lstinline{P[i]}
\end{itemize}

\begin{lstlisting}[aboveskip=1pc,xleftmargin=6pc]
k m
P[0]
P[1]    <%\SuppressNumber\AlternateNumber{...}%>
        <%\AlternateNumber{m+1}%>
P[m-1]  <%\ReactivateNumber%>
\end{lstlisting}

\noindent
โปรแกรมที่เขียนขึ้นจะต้องคืนค่า Sequence \lstinline{S} เป็น Output Data (ซึ่งเป็นคำตอบตามที่ระบุไว้ในหัวข้อ Objectives ข้างต้น) ผ่าน Standard Output ซึ่งมีรูปแบบดังต่อไปนี้

\begin{itemize}
    \item
        บรรทัดแรกมีจำนวนเต็มหนึ่งจำนวน \lstinline{n}
    \item 
        บรรทัดที่ \lstinline{j+2} 
        สำหรับ \lstinline{j} \lstinline{=} \lstinline{0}, \lstinline{1}, \ldots, \lstinline{n-1}
        จะมีจำนวนเต็มหนึ่งจำนวน ซึ่งก็คือ \lstinline{S[j]}
\end{itemize}

\begin{lstlisting}[aboveskip=1pc,xleftmargin=6pc]
n
S[0]
S[1]    <%\SuppressNumber\AlternateNumber{...}%>
        <%\AlternateNumber{n+1}%>
S[n-1]  <%\ReactivateNumber%>
\end{lstlisting}

\begin{center}
\smallskip\small
\begin{tabular}{p{0.425\linewidth}p{0.425\linewidth}}
\toprule
Example Input & Example Output \\
\midrule
\ttfamily\setSpacing{1}
5 5 \newline
1 \newline
3 \newline
5 \newline
2 \newline
3 &
\ttfamily\setSpacing{1}
8 \newline
1 \newline
3 \newline
5 \newline
2 \newline
3 \newline
1 \newline
4 \newline
1 \\
\bottomrule
\end{tabular}
\end{center}


\setSpacing{1.5}
\subsection{Scoring}

\noindent
โปรแกรมของคุณจะถูกทดสอบกับ Test Cases ที่มีเงื่อนไขต่าง ๆ ดังนี้

\begin{description}
    \item[SMALL] (คะแนน 20\%) \\
        รับประกันว่าความยาวของ Input Sequence จะสอดคล้องกับเงื่อนไข \lstinline{1 ≤ m ≤ 100} \\ และสมาชิกของ Irregular Sequence จะมี Upper Bound ที่สอดคล้องกับเงื่อนไข \lstinline{2 ≤ k ≤ 100}
    \item[MEDIUM] (คะแนน 35\%) \\
        รับประกันว่าความยาวของ Input Sequence จะสอดคล้องกับเงื่อนไข \lstinline{1 ≤ m ≤ 10,000} \\ และสมาชิกของ Irregular Sequence จะมี Upper Bound ที่สอดคล้องกับเงื่อนไข \lstinline{2 ≤ k ≤ 10,000}
    \item[LARGE] (คะแนน 45\%) \\
        ไม่มีเงื่อนไขเพิ่มเติม
\end{description}


\setSpacing{1.5}
\subsection{Limitations}

\noindent
โปรแกรมจะถูกจำกัดเวลาอยู่ที่ 0.6 วินาทีต่อ Test Case (baseline) และถูกจำกัดหน่วยความจำอยู่ที่ 512 MB
\begin{itemize}
    \item 
        สำหรับโปรแกรมที่เขียนด้วยภาษา C หรือ C++ จะถูกจำกัดเวลาเท่ากับค่า baseline ข้างต้น
    \item 
        สำหรับโปรแกรมที่เขียนด้วยภาษา Go หรือ Java จะถูกจำกัดเวลาอยู่ที่ 1.5 เท่าของ baseline ข้างต้น
    \item 
        สำหรับโปรแกรมที่เขียนด้วยภาษา JavaScript หรือ Python จะถูกจำกัดเวลาอยู่ที่ 2.5 เท่าของ baseline ข้างต้น
\end{itemize}
