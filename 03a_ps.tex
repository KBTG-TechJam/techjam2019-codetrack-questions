%!TEX root = book.tex
\section{Protein Synthesis}

\setSpacing{1.5}
\subsection{Problem Statement}

ณ ศูนย์วิจัยทางวิทยาศาสตร์ชีวโมเลกุลแห่งหนึ่ง มีนักวิจัยที่กำลังทำวิจัยเรื่องสมบัติของ Mixture (สารผสม) ของสาย Polypeptide หลากหลายประเภท \;
โดยปกติแล้วสาย Polypeptide จะประกอบไปด้วยกรดอะมิโนอย่างน้อย 1 ตัวถูกจับมาเรียงตัวกันเป็นเส้นตรงผ่านพันธะเอไมด์ (Amide Bond) \;
แต่สำหรับโจทย์ในข้อนี้ เราจะสนใจกรดอะมิโนเพียงสองชนิด ได้แก่ Serine (\lstinline{'S'}) และ Threonine (\lstinline{'T'}) 

ในการเตรียมสารนี้ นักวิจัยต้องการให้สารดังกล่าวเป็น Mixture ของสาย Polypeptide ทั้งหมด \lstinline{n} ชนิด \;
โดยสาย Polypeptide สองประเภทใด ๆ ไม่ควรเป็น Prefix ต่อกันและกัน (เพราะความคล้ายคลึงกันของสาย Polypeptide แบบดังกล่าวอาจจะทำให้การวัดอ่านค่าในการทดลองมีความสับสนได้) เช่น

\begin{itemize}
\item  สาย Polypeptide \lstinline{A = "STSST"} และ \lstinline{B = "STS"} ไม่ควรนำมาผสมกันใน Mixture เพราะ \lstinline{B} เป็น Prefix ของ \lstinline{A}
\item  สาย Polypeptide \lstinline{A = "TTS"} และ \lstinline{B = "TSSTTS"} สามารถนำมาผสมอยู่ใน Mixture เดียวกันได้เพราะไม่มีสาย Polypeptide ใดที่เป็น Prefix ของ Polypeptide อีกสายหนึ่ง
\end{itemize}

การสังเคราะห์สาย Polypeptide แต่ละประเภทจะมีต้นทุนที่ต่างกัน \;
กล่าวคือในการสร้างสาย Polypeptide หนึ่งจะต้องใช้ต้นทุน \lstinline{Cs} หน่วยต่อ Serine หนึ่งตัวที่ปรากฏในสาย Polypeptide ดังกล่าว และจะต้องใช้ต้นทุนอีก \lstinline{Ct} หน่วยต่อ Threonine หนึ่งตัวที่ปรากฏในสาย Polypeptide ดังกล่าว \;
เช่น การสร้างสาย Polypeptide \lstinline{"STTSSTT"} จะมีต้นทุนเท่ากับ $3 \text{\lstinline{Cs}} + 4 \text{\lstinline{Ct}}$ หน่วย

อยากทราบว่าเราจะสามารถสังเคราะห์สาย Polypeptide ทั้งหมด \lstinline{n} ชนิดจากกรดอะมิโนสองชนิดข้างต้น โดยที่ไม่มีสาย Polypeptide สองประเภทใด ๆ ที่เป็น Prefix ต่อกันและกัน เพื่อมาผสมกันเป็น Mixture ที่ต้องการ โดยใช้ต้นทุนเป็นมูลค่าต่ำที่สุดเท่าใด


\setSpacing{1.5}
\subsection{Objectives}

\noindent
จงเขียนโปรแกรมเพื่อรับ Input Data ดังต่อไปนี้
\begin{itemize}
\item
    จำนวนสาย Polypeptide \lstinline{n} ประเภทที่ต้องการสังเคราะห์ (โดยที่ \lstinline{1 ≤ n ≤ 100,000,000})
\item 
    ราคาของกรดอะมิโนต่อตัวสำหรับ Serine (\lstinline{Cs}) และ Threonine (\lstinline{Ct}) \\
    (โดยที่ \lstinline{0 ≤ Cs, Ct ≤ 100,000})
\end{itemize}

\noindent
แล้วจึงคำนวณหา\uline{ต้นทุนที่ต่ำที่สุด}สำหรับสร้างสาย Polypeptide ทั้งหมด \lstinline{n} ประเภทที่ไม่เป็น Prefix ต่อกันและกัน ตามราคาของกรดอะมิโนที่กำหนดเพื่อผสมเป็น Mixture


\newpage
\setSpacing{1.5}
\subsection{Interfaces and Data Format}

\noindent
โปรแกรมที่เขียนขึ้นจะต้องรับ Input Data ผ่าน Standard Input ซึ่งมีรูปแบบดังต่อไปนี้

\begin{itemize}
    \item
        บรรทัดเดียวมีจำนวนเต็มสามจำนวนคือ \lstinline{n} \lstinline{Cs} และ \lstinline{Ct} คั่นด้วยช่องว่าง
\end{itemize}

\begin{lstlisting}[aboveskip=1pc,xleftmargin=6pc]
n Cs Ct
\end{lstlisting}

โปรแกรมที่เขียนขึ้นจะต้องคืน Output Data ผ่าน Standard Output ซึ่งเป็นมูลค่าที่ตำที่สุดที่สามารถใช้ผสม Mixture ตามที่กำหนดใน Objectives ข้างต้น

\begin{center}
\smallskip\small
\begin{tabular}{p{0.425\linewidth}p{0.425\linewidth}}
\toprule
Example Input & Example Output \\
\midrule
\ttfamily\setSpacing{1}
4 1 2 &
\ttfamily\setSpacing{1}
12 \\
\bottomrule
\end{tabular}
\end{center}

\medskip
\textbf{หมายเหตุ:} ในตัวอย่างข้างต้น เราสามารถสร้างสาย Polypeptide \lstinline{"SS"} (ต้นทุน 2 หน่วย), \lstinline{"ST"} (ต้นทุน 3 หน่วย), \lstinline{"TS"} (ต้นทุน 3 หน่วย), และ \lstinline{"TT"} (ต้นทุน 4 หน่วย) ได้ \;
ดังนั้นแล้วต้นทุนรวมของ Mixture นี้เท่ากับ $2 + 3 + 3 + 4 = 12$ หน่วย ซึ่งน้อยที่สุดที่เป็นไปได้ในกรณีนี้ 


\setSpacing{1.5}
\subsection{Scoring}

\noindent
โปรแกรมของคุณจะถูกทดสอบกับ Test Cases ที่มีเงื่อนไขต่าง ๆ ดังนี้

\begin{description}[topsep=0pc,itemsep=0pc]
    \item[SMALL] (คะแนน 20\%) \\
        รับประกันว่าจำนวนของประเภทสาย Polypeptide ที่ต้องการจะสอดคล้องกับเงื่อนไข \lstinline{1 ≤ n ≤ 100} และราคาของกรดอะมิโนต่อตัวในสาย Polypeptide จะสอดคล้องกับเงื่อนไข \lstinline{0 ≤ Cs, Ct ≤ 100}
    \item[MEDIUM] (คะแนน 35\%) \\
        รับประกันว่าจำนวนของประเภทสาย Polypeptide ที่ต้องการจะสอดคล้องกับเงื่อนไข \lstinline{1 ≤ n ≤ 100,000}
    \item[LARGE] (คะแนน 45\%) \\
        ไม่มีเงื่อนไขเพิ่มเติม
\end{description}


\setSpacing{1.5}
\subsection{Limitations}

\noindent
โปรแกรมจะถูกจำกัดเวลาอยู่ที่ 0.4 วินาทีต่อ Test Case (baseline) และถูกจำกัดหน่วยความจำอยู่ที่ 512 MB
\begin{itemize}[topsep=0pc,itemsep=0pc]
    \item 
        สำหรับโปรแกรมที่เขียนด้วยภาษา C หรือ C++ จะถูกจำกัดเวลาเท่ากับค่า baseline ข้างต้น
    \item 
        สำหรับโปรแกรมที่เขียนด้วยภาษา Go หรือ Java จะถูกจำกัดเวลาอยู่ที่ 1.5 เท่าของ baseline ข้างต้น
    \item 
        สำหรับโปรแกรมที่เขียนด้วยภาษา JavaScript หรือ Python จะถูกจำกัดเวลาอยู่ที่ 2.5 เท่าของ baseline ข้างต้น
\end{itemize}
