%!TEX root = book.tex
\section{Sensors and Radio Tower}

\setSpacing{1.5}
\subsection{Problem Statement}

สำนักสำรวจวนอุทยาน (State Forest Survey Bereau) ต้องการทราบข้อมูลทางอุตุนิยมวิทยา (จำพวกเช่น อุณหภูมิ ความชื้น ความกดอากาศ ทิศทางลม ปริมาณฝุ่น PM 2.5 เป็นต้น) ทางสำนักสำรวจจึงมีโครงการที่จะติดตั้ง Sensor (มาตรวัดข้อมูล) ที่ตำแหน่งต่าง ๆ ภายในวนอุทยานแห่งหนึ่งซึ่งเป็นพื้นราบสองมิติ \;
และเพื่อความสะดวกรวดเร็วในการเก็บเกี่ยวข้อมูลจาก Sensor ที่ติดตั้งนั้น สำนักสำรวจยังจะติดตั้ง Radio Tower (เสาสื่อสัญญาณ) 1 ต้น ณ ตำแหน่งใดตำแหน่งหนึ่งภายในวนอุทยานเดียวกันนี้ โดยอาจเป็นตำแหน่งเดียวกับ Sensor บางตัวก็ได้ \;

Radio Tower ต้นดังกล่าวนี้จำเป็นต้องใช้พลังในการทำงานเพื่อรับและส่งสัญญาณกับ Sensor ที่ติดตั้งไว้ทุกตัวอย่างต่อเนื่องตลอดเวลา \;
\uline{ในแต่ละวินาที}นั้น Radio Tower ซึ่งติดตั้งอยู่ที่ตำแหน่ง $t = (x^*, y^*)$ จะต้องใช้พลังงานติดต่อกับ Sensor หนึ่งตัวที่ติดตั้งอยู่ที่ตำแหน่ง $s_i = (x_i, y_i)$ ซึ่งสอดคล้องกับสมการต่อไปนี้
\[
    e(t, s_i) := \text{พลังงานที่ใช้ต่อวินาที}(t, s_i) = \max\big\{ |x^* - x_i|, |y^* - y_i| \big\}
\]

\noindent
และพลังงานทั้งหมดที่ Radio Tower ดังกล่าวต้องการใช้ในหนึ่งวินาที จะมีค่าเท่ากับผลรวมของพลังงานที่ Radio Tower ใช้ติดต่อไปยัง Sensor $i \in S$ ทุกตัวที่ถูกติดตั้ง ซึ่งสามารถสรุปความได้ด้วยสมการดังต่อไปนี้
\[
    E(t) := \text{พลังงานรวมต่อวินาที(t)} = \sum_{i \,\in\, S} \,\text{พลังงานที่ใช้ต่อวินาที}(t, s_i) = \sum_{i \,\in\, S} e(t, s_i) 
\]

ในช่วงเริ่มต้นโครงการ ทางสำนักสำรวจยังไม่แน่ใจว่าควรติดตั้ง Sensor ณ ตำแหน่งใดบ้าง สำนักสำรวจจึงวางแผนโครงการตามลำดับดังนี้

\begin{enumerate}
    \item
        สำนักสำรวจเลือกติดตั้ง Sensor ตัวแรก ({\hrsp}$i = 1${\hrsp}) ณ ตำแหน่ง $s_1 = (x_1, y_1)$
    \item
        จากนั้นสำนักสำรวจจึงนำ Radio Tower ไปวางในตำแหน่งที่ใช้พลังงานรวมต่อวินาทีต่ำที่สุด 
        (ซึ่งในตอนนี้จะเป็นตำแหน่งเดียวกับ Sensor ตัวแรก ซึ่งจะทำให้ใช้พลังงาน 0 หน่วยต่อวินาที)
    \item  
        เมื่อสำนักสำรวจทราบข้อมูลเพิ่มเติมแล้ว สำนักสำรวจจะกำหนดตำแหน่งที่จะติดตั้ง Sensor ตัวที่ $i = 2, 3, \ldots$ ต่อไป \;
        และทุก ๆ ครั้งที่มีการติดตั้ง Sensor ตัวใหม่ตัวที่ ทางสำนักสำรวจอาจจำเป็นต้องย้ายตำแหน่งที่ตั้งของ Radio Tower ไปยังตำแหน่งใหม่เพื่อให้ได้พลังงานรวมต่อวินาทีค่าใหม่ที่ต่ำที่สุด
\end{enumerate}

จงคำนวณหาตำแหน่งที่ควรติดตั้ง Radio Tower ที่จะทำให้ค่าพลังงานรวมต่อวินาทีที่ต่ำที่สุดที่เพียงพอที่จะให้ Radio Tower ทำงานได้ หลังจากที่ติดตั้ง Sensor \uline{แต่ละตัว} (ตัวที่ $i = 1, 2, \ldots, n$) เป็นที่เรียบร้อยแล้ว \;
หากมีตำแหน่งที่สามารถติดตั้ง Radio Tower ได้หลายตำแหน่ง ให้ตอบตำแหน่งใดก็ได้


\newpage
\setSpacing{1.5}
\subsection{Situation Example}

\noindent
จงพิจารณาเหตุการณ์ที่เกิดขึ้นต่อไปนี้

\begin{enumerate}
    \item
        สำนักสำรวจเลือกติดตั้ง Sensor ตัวที่ $i = 1$ ณ ตำแหน่ง $s_1 = (1, 2)$

        ในกรณีนี้ตำแหน่งที่ดีที่สุดสำหรับ Radio Tower ตำแหน่งหนึ่งที่เป็นไปได้คือ ณ ตำแหน่ง $t = (1, 2)$ ซึ่งเป็นตำแหน่งเดียวกับ Sensor ตัวเดียวที่มีอยู่ \; พลังงานที่ใช้งานสำหรับแผนนี้คือ 0 หน่วยต่อวินาที
    \item
        สำนักสำรวจเลือกติดตั้ง Sensor ตัวที่ $i = 2$ ณ ตำแหน่ง $s_2 = (1, 5)$

        ในกรณีนี้ตำแหน่งที่ดีที่สุดสำหรับ Radio Tower ตำแหน่งหนึ่งที่เป็นไปได้คือ ณ ตำแหน่ง $t = (1, 2)$ ซึ่งเป็นตำแหน่งเดิมของ Radio Tower ไม่เปลี่ยนแปลง \; พลังงานที่ใช้งานต่อวินาทีสำหรับแผนนี้คือ
        \begin{align*}
            E(t) &= e(t, s_1) + e(t, s_2) \\
            e(t, s_1) &= \max\{|1 - 1|, |2 - 2|\} = \max\{0, 0\} = 0 \\
            e(t, s_2) &= \max\{|1 - 1|, |2 - 5|\} = \max\{0, 3\} = 3 \\
            \therefore\quad E(t) &= 0 + 3 = 3
        \end{align*}
        ซึ่งมีค่าต่ำที่สุดเท่าที่เป็นไปได้
    \item
        สำนักสำรวจเลือกติดตั้ง Sensor ตัวที่ $i = 3$ ณ ตำแหน่ง $s_3 = (5, 7)$

        ในกรณีนี้ตำแหน่งที่ดีที่สุดสำหรับ Radio Tower ตำแหน่งหนึ่งที่เป็นไปได้คือ ณ ตำแหน่ง $t = (2, 4)$ \; พลังงานที่ใช้งานต่อวินาทีสำหรับแผนนี้คือ
        \begin{align*}
            E(t) &= e(t, s_1) + e(t, s_2) + e(t, s_3) \\
            e(t, s_1) &= \max\{|2 - 1|, |4 - 2|\} = \max\{1, 2\} = 2 \\
            e(t, s_2) &= \max\{|2 - 1|, |4 - 5|\} = \max\{1, 1\} = 1 \\
            e(t, s_3) &= \max\{|2 - 5|, |4 - 7|\} = \max\{3, 3\} = 3 \\
            \therefore\quad E(t) &= 2 + 1 + 3 = 6
        \end{align*}
        ซึ่งมีค่าต่ำที่สุดเท่าที่เป็นไปได้
    \item
        สำนักสำรวจเลือกติดตั้ง Sensor ตัวที่ $i = 4$ ณ ตำแหน่ง $s_4 = (7, 1)$

        ในกรณีนี้ตำแหน่งที่ดีที่สุดสำหรับ Radio Tower ตำแหน่งหนึ่งที่เป็นไปได้คือ ณ ตำแหน่ง $t = (2, 4)$ เช่นเดิม \; พลังงานที่ใช้งานต่อวินาทีสำหรับแผนนี้คือ
        \begin{align*}
            E(t) &= e(t, s_1) + e(t, s_2) + e(t, s_3) + e(t, s_4) \\
            e(t, s_1) &= \max\{|2 - 1|, |4 - 2|\} = \max\{1, 2\} = 2 \\
            e(t, s_2) &= \max\{|2 - 1|, |4 - 5|\} = \max\{1, 1\} = 1 \\
            e(t, s_3) &= \max\{|2 - 5|, |4 - 7|\} = \max\{3, 3\} = 3 \\
            e(t, s_3) &= \max\{|2 - 7|, |4 - 1|\} = \max\{5, 3\} = 5 \\
            \therefore\quad E(t) &= 2 + 1 + 3 + 5 = 11
        \end{align*}
        ซึ่งมีค่าต่ำที่สุดเท่าที่เป็นไปได้
\end{enumerate}


\setSpacing{1.5}
\subsection{Objectives}

จงเขียนโปรแกรมเพื่อรับ Input Data เป็นพิกัดตำแหน่งที่ติดตั้ง Sensor รวมทั้งสิ้น $n$ ตัว \; 
หลังจากที่โปแกรมของผู้เข้าแข่งขันพิจารณาข้อมูลตำแหน่งของ Sensor แต่ละตัวตามลำดับแล้ว ให้ตอบ\uline{พิกัดของ Radio Tower ที่ใช้พลังงานรวมต่อวินาทีน้อยที่สุด}เป็น Output Data ก่อนที่จะพิจารณา Sensor ตัวถัดไป \;

จำนวน Sensor ทั้งหมดที่จะติดตั้งในวนอุทยานจะสอดคล้องกับเงื่อนไข $1 \leq n \leq 200,\!000$ นอกจากนั้น ค่าพิกัดที่ปรากฏใน Input Data ทั้งหมดจะเป็นข้อมูลจุดพิกัด $(x,y)$ บนระนาบ 2 มิติ ซึ่งสอดคล้องกับเงื่อนไข $-10^9 \leq x, y \leq 10^9$


\setSpacing{1.5}
\subsection{Interfaces and Data Format}

\noindent
โปรแกรมที่เขียนขึ้นจะต้องรับ Input Data ผ่าน Standard Input ดังต่อไปนี้

\begin{itemize}
    \item
        บรรทัดแรกมีจำนวนเต็มหนึ่งจำนวนคือ $n$ 
    \item 
        บรรทัดที่ $i+1$ สำหรับ $i = 1, 2, \ldots, n$ จะมีจำนวนเต็มสองจำนวน \\
        ซึ่งก็คือ $x_i$ และ $y_i$ ซึ่งระบุพิกัดของ Sensor ตัวที่ $i$
\end{itemize}

\begin{lstlisting}[aboveskip=1pc,xleftmargin=6pc,mathescape=true]
$n$ 
$x_1$ $y_1$
$x_2$ $y_2$     <%\SuppressNumber\AlternateNumber{...}%>
                <%\AlternateNumber{n+1}%>
$x_n$ $y_n$     <%\ReactivateNumber%>
\end{lstlisting}

\noindent
โปรแกรมที่เขียนขึ้นจะต้องคืน Output Data ผ่าน Standard Output ซึ่งประกอบด้วยข้อมูลดังต่อไปนี้

\begin{itemize}
    \item 
        บรรทัดที่ \,$i$\, สำหรับ \,$i = 1, 2, \ldots, n$\, จะมีจำนวนสองจำนวน ซึ่งก็คือ $x^*_i$ และ $y^*_i$ ซึ่งระบุพิกัดของ Radio Tower ที่ใช้พลังงานรวมต่อวินาทีในการสื่อสารกับ Sensor ตัวที่ $1$ ถึง $i$ น้อยที่สุด \;
       ค่าพิกัด $x^*_i$ และ $y^*_i$ จะต้องเป็น\uline{จำนวนเต็มหรือจำนวนจริง}ที่มีตัวเลขหลังจุดทศนิยมไม่เกิน 2 ตำแหน่ง
\end{itemize}

\begin{lstlisting}[aboveskip=1pc,xleftmargin=6pc,mathescape=true]
$x^*_1$ $y^*_1$
$x^*_2$ $y^*_2$     <%\SuppressNumber\AlternateNumber{...}%>
                    <%\AlternateNumber{n}%>
$x^*_n$ $y^*_n$     <%\ReactivateNumber%>
\end{lstlisting}

\begin{center}
\smallskip\small
\begin{tabular}{p{0.425\linewidth}p{0.45\linewidth}}
\toprule
Example Input & Example Output \\
\midrule
\ttfamily\setSpacing{1}
4 \newline
1 2 \newline
1 5 \newline
5 7 \newline
7 1 &
\ttfamily\setSpacing{1}
1 2 \newline
1 2 \newline
2 4 \newline
2 4 \\
\bottomrule
\end{tabular}
\end{center}


\newpage
\setSpacing{1.5}
\subsection{Scoring}

\noindent
โปรแกรมของคุณจะถูกทดสอบกับ Test Cases ที่มีเงื่อนไขต่าง ๆ ดังนี้

\begin{description}
    \item[SMALL] (คะแนน 20\%) \\
        รับประกันว่าจำนวน Sensor ทั้งหมดจะสอดคล้องกับเงื่อนไข $1 \leq n \leq 2,\!000$ และค่าพิกัด $(x, y)$ ของ Sensor ที่จะติดตั้งทุกตัวจะสอดคล้องกับเงื่อนไข $-100 \leq x, y \leq 100$
    \item[MEDIUM] (คะแนน 35\%) \\
        รับประกันว่าจำนวน Sensor ทั้งหมดจะสอดคล้องกับเงื่อนไข $1 \leq n \leq 2,\!000$
    \item[LARGE] (คะแนน 45\%) \\
        ไม่มีเงื่อนไขเพิ่มเติม
\end{description}


\setSpacing{1.5}
\subsection{Limitations}

\noindent
โปรแกรมจะถูกจำกัดเวลาอยู่ที่ 0.8 วินาทีต่อ Test Case (baseline) และถูกจำกัดหน่วยความจำอยู่ที่ 512 MB
\begin{itemize}
    \item 
        สำหรับโปรแกรมที่เขียนด้วยภาษา C หรือ C++ จะถูกจำกัดเวลาเท่ากับค่า baseline ข้างต้น
    \item 
        สำหรับโปรแกรมที่เขียนด้วยภาษา Go หรือ Java จะถูกจำกัดเวลาอยู่ที่ 1.5 เท่าของ baseline ข้างต้น
    \item 
        สำหรับโปรแกรมที่เขียนด้วยภาษา JavaScript หรือ Python จะถูกจำกัดเวลาอยู่ที่ 2.5 เท่าของ baseline ข้างต้น
\end{itemize}