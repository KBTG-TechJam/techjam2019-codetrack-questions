%!TEX root = book.tex
\section{Reconstruct Transaction List}

\setSpacing{1.5}
\subsection{Problem Statement}

นายกสิกรเป็นคนที่ชอบจดบันทึกรายรับ-รายจ่ายของตนเอง \; 
ในตอนต้นเดือนของทุก ๆ เดือนเขาจะนำเงินที่เหลือค้างจากเดือนก่อนหน้าออกจากกระเป๋าตังค์ของเขาทั้งหมด และเริ่มต้นเดือนใหม่ด้วยเงิน 0 บาทในกระเป๋าตังค์ \;
นอกจากนั้น เขาจะนำสมุดเปล่าสำหรับจดบันทึกรายการรายรับ-รายจ่ายใส่ไว้ในกระเป๋าเป้ไว้ 2 เล่ม เพื่อเอาไว้บันทึกรายรับ-รายจ่ายสำหรับเดือนใหม่

ตลอดทั้งเดือนนี้ ทุก ๆ ครั้งที่นายกสิกรมีเงินเข้าหรือออกจากกระเป๋าตังค์ (กล่าวคือมี \textbf{Transaction} รายการใหม่เกิดขึ้น) \;เขาจะสุ่มเลือกหยิบสมุดบันทึกรายการรายรับ-รายจ่ายจากกระเป๋าเป้ออกมา 1 เล่ม \;
แล้วจากนั้นเขาจะจดบันทึก Transaction ดังกล่าวต่อท้ายรายการก่อนหน้าที่เคยจดไว้แล้วในสมุดเล่มนั้น \;

เมื่อถึงเวลาสิ้นเดือน นายกสิกรจะนำบันทึกรายการ Transaction จากสมุดทั้งสองเล่มมาเขียนรวมกันกลายเป็นบันทึกรายการรายรับ-รายจ่ายฉบับเต็ม (\textbf{Consolidated Transaction List}) โดยที่เงื่อนไขดังต่อไปนี้ \;

\begin{enumerate}
    \item ลำดับของ Transaction ที่ปรากฏในบันทึกฉบับเต็ม จะต้องสอดคล้องกับลำดับของ Transaction ที่ปรากฏในสมุดแต่ละเล่ม
    \item เมื่อพิจารณารายการรายรับ-รายจ่าย จากรายการแรกไปยังรายการสุดท้าย ยอดคงเหลือสะสม ณ เวลาใด ๆ ไม่สามารถมีค่าติดลบได้ (นั่นก็เพราะว่าไม่มีเวลาใด ๆ ที่นายกสิกรใช้จ่ายเงินมากกว่าเงินที่คงเหลือในกระเป๋าตังค์ของเขา)
\end{enumerate}


\setSpacing{1.5}
\subsection{Situation Example}

\noindent
สมมติว่า \lstinline{A} และ \lstinline{B} คือลำดับของ Transaction ที่ปรากฏภายในสมุดทั้งสองเล่มตามลำดับ ซึ่งมีค่าดังต่อไปนี้

\begin{lstlisting}[xleftmargin=2pc,numbers=none]
A = [10, -7, -4]
B = [-8, 6, 8, -4]
\end{lstlisting}

\noindent
รายการ \lstinline{A} และ \lstinline{B} ข้างต้นสามารถนำมารวมกันเป็น Consolidated Transaction List ได้ 5 วิธี ได้แก่

\begin{itemize}[leftmargin=2pc,topsep=0.25pc] 
    \setSpacing{1}
    \item[] \textbf{วิธีที่ 1.}\;\; \lstinline{[ 10 A, -8 B, 6 B, -7 A,  8 B, -4 A, -4 B ]}
    \item[] \textbf{วิธีที่ 2.}\;\; \lstinline{[ 10 A, -8 B, 6 B, -7 A,  8 B, -4 B, -4 A ]}
    \item[] \textbf{วิธีที่ 3.}\;\; \lstinline{[ 10 A, -8 B, 6 B,  8 B, -7 A, -4 A, -4 B ]}
    \item[] \textbf{วิธีที่ 4.}\;\; \lstinline{[ 10 A, -8 B, 6 B,  8 B, -7 A, -4 B, -4 A ]}
    \item[] \textbf{วิธีที่ 5.}\;\; \lstinline{[ 10 A, -8 B, 6 B,  8 B, -4 B, -7 A, -4 A ]}
\end{itemize}


\newpage
\setSpacing{1.5}
\subsection{Objectives}

\noindent
จงเขียนโปรแกรมเพื่อรับ Input Data ต่อไปนี้

\begin{itemize}[topsep=0pc,itemsep=0pt]
    \item 
        ลำดับของ Transaction \lstinline{A}\:
        (ประกอบด้วยจำนวนเต็ม \lstinline{A[0]}, \lstinline{A[1]}, \ldots, \lstinline{A[n-1]} โดยที่ \lstinline{1 ≤ n ≤ 2,000} และ Transaction \lstinline{A[i]} แต่ละรายการจะสอดคล้องกับเงื่อนไข \lstinline{-500,000 ≤ A[i] ≤ 500,000}) \;
        ซึ่งจะแสดงยอดรายรับหรือรายจ่ายแต่ละ Transaction ที่ถูกบันทึกไว้ภายในสมุดเล่มแรกในหนึ่งเดือน
    \item 
        ลำดับของ Transaction \lstinline{B}\:
        (ประกอบด้วยจำนวนเต็ม \lstinline{B[0]}, \lstinline{B[1]}, \ldots, \lstinline{B[m-1]} โดยที่ \lstinline{1 ≤ m ≤ 2,000} และ Transaction \lstinline{B[j]} แต่ละรายการจะสอดคล้องกับเงื่อนไข \lstinline{-500,000 ≤ B[j] ≤ 500,000}) \;
        ซึ่งจะแสดงยอดรายรับหรือรายจ่ายแต่ละ Transaction ที่ถูกบันทึกไว้ภายในสมุดเล่มที่สองในเดือนเดียวกัน
\end{itemize}

\noindent
แล้วจึงคำนวณ\uline{จำนวนวิธี}ที่นายกสิกรสามารถควบรวมบันทึกรายการ Transaction จากสมุดทั้งสองเล่มให้กลาย Consolidated Transaction List ตอนท้ายเดือน และคืนค่าคำตอบดังกล่าวเป็น Output Data ของโปรแกรม

หากคำตอบข้างต้นมีค่ามากกว่าหรือเท่ากับ 1,000,000,007 ให้ตอบคำตอบในรูปเศษที่เกิดจากการหารค่าด้งกล่าวด้วยจำนวนเต็ม 1,000,000,007


\setSpacing{1.5}
\subsection{Interfaces and Data Format}

\noindent
โปรแกรมที่เขียนขึ้นจะต้องรับ Input Data ผ่าน Standard Input ซึ่งมีรูปแบบดังต่อไปนี้

\begin{itemize}[topsep=0pc,itemsep=0pt]
    \item
        บรรทัดแรกมีจำนวนเต็มหนึ่งจำนวน ซึ่งก็คือ \lstinline{n}
    \item 
        บรรทัดที่ \lstinline{i+2} 
        สำหรับ \lstinline{i} \lstinline{=} \lstinline{0}, \lstinline{1}, \ldots, \lstinline{n-1}
        จะมีจำนวนเต็มหนึ่งจำนวน ซึ่งก็คือ \lstinline{A[i]}
    \item
        บรรทัดที่ \lstinline{n+2} มีจำนวนเต็มหนึ่งจำนวน ซึ่งก็คือ \lstinline{m}
    \item 
        บรรทัดที่ \lstinline{n+j+3} 
        สำหรับ \lstinline{j} \lstinline{=} \lstinline{0}, \lstinline{1}, \ldots, \lstinline{m-1}
        จะมีจำนวนเต็มหนึ่งจำนวน ซึ่งก็คือ \lstinline{B[j]}
\end{itemize}

\begin{lstlisting}[aboveskip=1pc,xleftmargin=6pc]
n
A[0]
A[1]    <%\SuppressNumber\AlternateNumber{...}%>
        <%\AlternateNumber{n+1}%>
A[n-1]  <%\AlternateNumber{n+2}%>
m       <%\AlternateNumber{n+3}%>
B[0]    <%\AlternateNumber{n+4}%>
B[1]    <%\AlternateNumber{...}%>
        <%\AlternateNumber{n+m+2}%>
B[m-1]  <%\ReactivateNumber%>
\end{lstlisting}

\noindent
โปรแกรมที่เขียนขึ้นจะต้องคืน Output Data ผ่าน Standard Output เป็นจำนวนเต็ม 1 จำนวน ซึ่งเป็นคำตอบของโจทย์ตามที่ระบุไว้ในหัวข้อ Objectives ข้างต้น

\newpage
\begin{center}
\smallskip\small
\begin{tabular}{p{0.425\linewidth}p{0.425\linewidth}}
\toprule
Example Input & Example Output \\
\midrule
\ttfamily\setSpacing{1}
3 \newline
10 \newline
-7 \newline
-4 \newline
4 \newline
-8 \newline
6 \newline
8 \newline
-4 &
\ttfamily\setSpacing{1}
5 \\
\bottomrule
\end{tabular}
\end{center}


\setSpacing{1.5}
\subsection{Scoring}

\noindent
โปรแกรมของคุณจะถูกทดสอบกับ Test Cases ที่มีเงื่อนไขต่าง ๆ ดังนี้

\begin{description}[topsep=0pc,itemsep=0pc]
    \item[SMALL] (คะแนน 20\%) \\
        รับประกันว่าความยาวของ Input Transaction List แต่ละอันจะสอดคล้องกับเงื่อนไข \lstinline{1 ≤ n,m ≤ 100} \\ 
        และ Transaction แต่ละรายการจะสอดคล้องกับเงื่อนไข \lstinline{-1 ≤ A[i] ≤ 1} สำหรับ \lstinline{i} \lstinline{=} \lstinline{0}, \lstinline{1}, \ldots, \lstinline{n-1} และ \lstinline{-1 ≤ B[j] ≤ 1} สำหรับ \lstinline{j} \lstinline{=} \lstinline{0}, \lstinline{1}, \ldots, \lstinline{m-1} 
    \item[MEDIUM] (คะแนน 35\%) \\
        รับประกันว่าความยาวของ Input Transaction List แต่ละอันจะสอดคล้องกับเงื่อนไข \lstinline{1 ≤ n,m ≤ 100}
    \item[LARGE] (คะแนน 45\%) \\
        ไม่มีเงื่อนไขเพิ่มเติม
\end{description}


\setSpacing{1.5}
\subsection{Limitations}

\noindent
โปรแกรมจะถูกจำกัดเวลาอยู่ที่ 1.0 วินาทีต่อ Test Case (baseline) และถูกจำกัดหน่วยความจำอยู่ที่ 512 MB
\begin{itemize}[topsep=0pc,itemsep=0pc]
    \item 
        สำหรับโปรแกรมที่เขียนด้วยภาษา C หรือ C++ จะถูกจำกัดเวลาเท่ากับค่า baseline ข้างต้น
    \item 
        สำหรับโปรแกรมที่เขียนด้วยภาษา Go หรือ Java จะถูกจำกัดเวลาอยู่ที่ 1.5 เท่าของ baseline ข้างต้น
    \item 
        สำหรับโปรแกรมที่เขียนด้วยภาษา JavaScript หรือ Python จะถูกจำกัดเวลาอยู่ที่ 2.5 เท่าของ baseline ข้างต้น
\end{itemize}